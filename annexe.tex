\documentclass[14pt]{these}

\usepackage{mathpazo}
\renewcommand{\familydefault}{\rmdefault}
\usepackage[francais]{babel}
\usepackage{xkeyval}
\usepackage{graphicx}
\usepackage[utf8]{inputenc}
\usepackage[T1]{fontenc}
\usepackage{textcomp}
\usepackage{array}
\usepackage{longtable}
\usepackage{multirow}


\makeatletter
\define@cmdkey{bdd}[BDD]{house}{}
\define@cmdkey{bdd}[BDD]{name}{}
\define@cmdkey{bdd}[BDD]{subject}{}
\define@cmdkey{bdd}[BDD]{type}{}
\define@cmdkey{bdd}[BDD]{date}{}
\define@cmdkey{bdd}[BDD]{room}{}
\define@cmdkey{bdd}[BDD]{wall}{}
\define@cmdkey{bdd}[BDD]{artist}{}
\define@cmdkey{bdd}[BDD]{location}{}
\define@cmdkey{bdd}[BDD]{credits}{}
\define@cmdkey{bdd}[BDD]{material}{}
\define@cmdkey{bdd}[BDD]{discovery}{}
\define@cmdkey{bdd}[BDD]{ref}{}
\define@cmdkey{bdd}[BDD]{value}{}
\define@cmdkey{bdd}[BDD]{magistrate}{}
\define@cmdkey{bdd}[BDD]{front}{}
\define@cmdkey{bdd}[BDD]{frontInscr}{}
\define@cmdkey{bdd}[BDD]{back}{}
\define@cmdkey{bdd}[BDD]{backInscr}{}
\makeatother


\begin{document}

\part{D\'ecors}
\newcommand{\decor}[2][]{
  \setkeys{bdd}{#1}
  \centering
  \includegraphics[width=\textwidth,height=0.7\textheight,keepaspectratio]{#2}

  \BDDhouse\\
  \BDDname\\
  \textbf{Sujet:} \BDDsubject\\
  \textbf{Type:} \BDDtype\\ 
  \textbf{Datation:} \BDDdate\\ 
  \textbf{Pi\`ece, paroi:}\BDDroom, \BDDwall\\
  \textbf{Artiste:}\BDDartist\\ 
  \textbf{Localisation actuelle:}\BDDlocation\\
  \textbf{Cr\'edits photographiques:} \BDDcredits

  \clearpage
}
\input{annexe-decors.tex}

\subsection*{Nombres d'occurrences des héros homériques dans les décors de Pompéi
et d'Herculanum}

\begin{tabular}{|>{\centering}p{0.1\textwidth}|>{\centering}p{0.1\textwidth}|>{\centering}p{0.1\textwidth}|>{\centering}p{0.1\textwidth}|>{\centering}p{0.1\textwidth}|>{\centering}p{0.1\textwidth}|>{\centering}p{0.1\textwidth}|>{\centering}p{0.1\textwidth}|>{\centering}p{0.1\textwidth}|>{\centering}p{0.1\textwidth}|}
\hline 
\multicolumn{2}{|c|}{} & \multicolumn{3}{c|}{Espaces de réception} & \multicolumn{4}{c|}{Espaces privés} & \tabularnewline
\hline 
\hline 
Personnage & Sujet & \emph{Atrium} et ailes & \emph{Oecus} & \emph{Tablinum} & \emph{Triclinium} & \emph{Cubiculum, Diaeta} & Jardin & Péristyle & Autres\tabularnewline
\hline 
\multicolumn{2}{|c|}{\emph{Iliade} en entier} & I &  &  &  &  &  &  & I\tabularnewline
\hline 
\multicolumn{2}{|c|}{Chryséis (embarquement de)} & I &  &  &  &  &  &  & I\tabularnewline
\hline 
\multirow{7}{0.1\textwidth}{Achille} & et Chiron & 1 &  & 1 &  &  & 2 &  & I\tabularnewline
\cline{2-10} 
 & à Skyros & 1 &  & 1 & 2 &  & 1 & I & \tabularnewline
\cline{2-10} 
 & et Briséis & 2 &  &  &  &  &  &  & \tabularnewline
\cline{2-10} 
 & et Polyxène & 1 &  &  & 1 &  &  &  & \tabularnewline
\cline{2-10} 
 & tirant l'épée contre Agamemnon &  &  & 1 &  &  & 1 &  & \tabularnewline
\cline{2-10} 
 & et Télèphe &  &  &  &  & 1 &  &  & \tabularnewline
\cline{2-10} 
 & Jeux funéraires de Patrocle &  & 1 &  &  &  &  &  & \tabularnewline
\hline 
\multirow{2}{0.1\textwidth}{Thétis} & dans la forge d'Héphaïstos & 3 &  & 2 & 1 &  &  &  & \tabularnewline
\cline{2-10} 
 & apportant ses armes à Achille, sur un cheval marin &  & 1 &  & 2 &  &  & 2 & \tabularnewline
\hline 
\multicolumn{2}{|c|}{Jupiter et Junon (noces de)} & 1 &  &  &  &  &  &  & \tabularnewline
\hline 
\multirow{4}{0.1\textwidth}{Ulysse} & et le Palladion &  &  &  & 1 &  &  &  & \tabularnewline
\cline{2-10} 
 & et Circé & 1 &  &  &  &  &  &  & 1\tabularnewline
\cline{2-10} 
 & et Pénélope &  &  &  & 1 &  &  &  & 1\tabularnewline
\cline{2-10} 
 & et personnage non identifié & 1 &  &  &  &  &  &  & \tabularnewline
\hline 
\multirow{3}{0.1\textwidth}{Cassandre} & (viol de) & 1 & 1 &  &  &  &  &  & \tabularnewline
\cline{2-10} 
 & (prophétie de) &  &  &  & 2 &  &  &  & \tabularnewline
\cline{2-10} 
 & et le cheval de Troie & 1 &  &  &  &  &  &  & \tabularnewline
\hline 
\multirow{2}{0.1\textwidth}{Pâris} & (jugement de) &  &  &  & 6 & 1 &  & 1 & \tabularnewline
\cline{2-10} 
 & et Hélène &  &  & 1 & 1 & 4 &  &  & \tabularnewline
\hline 
\multirow{2}{0.1\textwidth}{Hélène} & et Pâris &  &  & 1 & 1 & 4 &  &  & \tabularnewline
\cline{2-10} 
 & et Ménélas & 1 &  &  & 1 &  &  &  & \tabularnewline
\hline 
\multirow{4}{0.1\textwidth}{Énée} & et Didon & 1 &  &  &  &  &  & 1 & \tabularnewline
\cline{2-10} 
 & blessé &  &  &  & 1 &  &  &  & \tabularnewline
\cline{2-10} 
 & et Polyphème &  &  & 1 &  &  &  &  & \tabularnewline
\cline{2-10} 
 & fuyant Troie &  &  &  &  &  &  &  & 1\tabularnewline
\hline 
Didon & abandonnée & 1 &  &  &  &  &  &  & \tabularnewline
\hline 
\multirow{2}{0.1\textwidth}{Iphigénie} & (sacrifice de) &  &  &  &  &  &  & 1 & 1\tabularnewline
\cline{2-10} 
 & en Tauride & 1 &  & 1 & 1 & 1 &  &  & 1\tabularnewline
\hline 
\multicolumn{2}{|c|}{Cheval de Troie} & 3 &  &  &  & 1 &  &  & \tabularnewline
\hline 
\multicolumn{2}{|c|}{Laocoon (mort de)} & 2 &  &  &  &  &  &  & \tabularnewline
\hline 
\multicolumn{2}{|c|}{Apollon et Poséidon devant Troie} & 1 &  &  &  &  &  & 1 & \tabularnewline
\hline 
\multicolumn{2}{|c|}{Ganymède} &  &  &  &  & 1 &  &  & \tabularnewline
\hline 
\end{tabular}

\subsection*{Œuvres à thématiques homériques regroupées par peintre, d'après le
catalogue de L. Richardson (Richardson 2000)}

\begin{longtable}{|>{\centering}p{0.2\textwidth}|>{\centering}p{0.2\textwidth}|>{\centering}p{0.3\textwidth}|>{\centering}p{0.3\textwidth}|}
\hline 
Peintre  & Maisons & Peintures à sujet homérique & Autres sujets représentés par le même peintre, dans la même maison. \tabularnewline
\hline 
\hline 
\multirow{3}{0.2\textwidth}{Le « peintre de Boscotrecase » (p. 36 \emph{sq.}) } & Maison du prêtre Amandus (I, 7, 7)  & \emph{Pâris présenté à Hélène} & \emph{Galatée et Polyphème, Hercule dans le jardin des Hespérides,
Chute d'Icare, Persée sauvant Andromède}\tabularnewline
\cline{2-4} 
 & Maison de Julius Polybius (IX, 13, 1-3)  & \emph{Origines de Rome } & \emph{Châtiment de Dircé}\tabularnewline
\cline{2-4} 
 & Maison de Siricus (VII, 1, 25-47)  & \emph{Énée blessé } & \tabularnewline
\hline 
Le peintre de Cassandre (p. 54 \emph{sq.})  & Maison de la Grille métallique (I, 2, 28)  & \emph{Prophétie de Cassandre } & \tabularnewline
\hline 
Le peintre de Caecilius Jucundus (p. 55 \emph{sq.})  & Maison de Caecilius Jucundus (V, 1, 26)  & \emph{Iphigénie en Tauride } & \emph{Ménade et amour, satyre et Ménade (trois exemplaires) }\tabularnewline
\hline 
\multirow{2}{0.2\textwidth}{Le peintre des Cinq Squelettes (p. 61 \emph{sq}.) } & Maison des Cinq Squelettes (VI, 10, 2)  & \emph{Pâris et Hélène, Prophétie de Cassandre, Ulysse et Pénélope } & \tabularnewline
\cline{2-4} 
 & Maison des Quadriges (VII, 2, 25)  & \emph{Forge de Vulcain } & \tabularnewline
\hline 
\multirow{3}{0.2\textwidth}{Le peintre du Cithariste (p. 62\emph{ sq}.) } & I, 2, 6  & \emph{Vol du Palladium} & \tabularnewline
\cline{2-4} 
 & Maison du Cithariste (I, 4, 5-25)  & \emph{Jugement de Pâris, Monarque oriental (Priam?) et messager } & \emph{Léda ou Némésis , Pindare et Corinne}\tabularnewline
\cline{2-4} 
 & Maison de Laocoon (VI, 14, 28-31)  & \emph{Mort de Laocoon, Énée et Polyphème } & \tabularnewline
\hline 
\multirow{2}{0.2\textwidth}{Le peintre de Jason (p. 68 \emph{sq}.) } & Maison des Amours Dorés (VI, 16, 7-38)  & \emph{Pâris et Hélène } & \emph{Jason et Pélias, sujet non identifié }\tabularnewline
\cline{2-4} 
 & Maison de Jason (IX, 5, 18)  & \emph{Pâris et Hélène, Phèdre et sa nourrice, Médée et ses enfants } & \emph{Rapt d'Europe, Satyre et nymphe, Hercule et Nessus, Sujet non
identifié, Jason et Pélias }\tabularnewline
\hline 
\multirow{2}{0.2\textwidth}{Le peintre d'Achille (p. 87 \emph{sq.})} & Maison du Cithariste (I, 4, 5-25)  & \emph{Iphigénie en Tauride } & \emph{Bacchus et Ariane }\tabularnewline
\cline{2-4} 
 & Maison des Dioscures (VI, 9, 6-7)  & \emph{Achille à Skyros, Achille et Agamemnon } & \tabularnewline
\hline 
\multirow{5}{0.2\textwidth}{Le peintre d'Adonis blessé (p. 91) } & Maison du Cithariste (I, 4, 5-25)  & \emph{Apollon et Poséidon devant Laomédon } & \emph{Ménade endormie, Mars et Vénus, Poète couronné de lierre}\tabularnewline
\cline{2-4} 
 & Maison de Caecilius Jucundus (V, 1, 26)  & \emph{Jugement de Pâris } & \emph{Thésée abandonne Ariane, Tondi de bustes }\tabularnewline
\cline{2-4} 
 & Maison d'Olconius Rufus (VIII, 4, 4)  & \emph{Jugement de Pâris Achille à Skyros } & \emph{Hermaphrodite et Silène, Bacchus et Ariane }\tabularnewline
\cline{2-4} 
 & Maison du Centenaire (IX, 8, 3-6)  & \emph{Iphigénie en Tauride } & \emph{Thésée et Minotaure, Hermaphrodite et Silène }\tabularnewline
\cline{2-4} 
 & Area sacra suburbana (Herculanum)  & \emph{Pâris et Hélène } & \emph{Silène faisant une libation}\tabularnewline
\hline 
Le peintre de la Maison des Guerriers (p. 104 \emph{sq.})  & Maison de Méléagre (VI, 9, 2-13)  & \emph{Thétis avec l'armure d'Achille } & \emph{Satyre, serpent et Ménade, Figures diverses}\tabularnewline
\hline 
\multirow{2}{0.2\textwidth}{Le peintre des Quatre Divinités (p. 115 \emph{sq}.)} & Maison des Uboni (IX, 5, 2)  & \emph{Achille à Skyros, Thétis apportant son armure, Thétis dans la
forge d'Héphaistos } & \tabularnewline
\cline{2-4} 
 & \emph{Taberna} (IX, 8, 5)  & \emph{Énée fuyant Troie, Romulus avec le trophée d'Acron } & \tabularnewline
\hline 
\multirow{4}{0.2\textwidth}{Le peintre d'Io (p. 122 \emph{sq}.) } & Maison de Pinarius Cerialis ou d'Iphigénie (III, 4, 4)  & \emph{Scaenae frons avec Iphigénie en Tauride } & \emph{Attis (?) et autres }\tabularnewline
\cline{2-4} 
 & Maison des Amours Dorés (VI, 16, 7-38)  & \emph{Thétis dans la forge d'Héphaïstos} & \tabularnewline
\cline{2-4} 
 & Macellum  & \emph{Pénélope et Ulysse } & \emph{Io et Argus, groupes et figures volants }\tabularnewline
\cline{2-4} 
 & Villa de San Marco  & \emph{Iphigénie tenant un xoanon } & \emph{Figures, Gorgoneion, Persée }\tabularnewline
\hline 
\multirow{2}{0.2\textwidth}{Le peintre d'Iphigénie (p. 129 \emph{sq.}), première période} & Maison du Ménandre (I \textendash{} 10, 4) & \emph{Mort de Laocoon, Cheval de Troie, viol de Cassandre } & \emph{Amours, Ménades et Bacchus enfant, Persée et Andromède, satyre
et amour, libération d'Andromède, Persée au palais de Céphée, punition
de Dircé, Muses }\tabularnewline
\cline{2-4} 
 & V, 2, 14  & \emph{Embarquement d'Hélène, Ulysse et Circé } & \tabularnewline
\hline 
période intermédiaire  & Maison du Poète Tragique (VI \textendash{} 8, 3-5) & \emph{Sacrifice d'Iphigénie, Énée et Didon } & \emph{Thésée abandonne Ariane, Admète recevant l'oracle, Saisons,
nid d'amours}\tabularnewline
\hline 
troisième période, travaux tardifs  & Maison du Diadumène (VII, 12, 26-27)  & \emph{Énée et Didon } & \emph{Thésée abandonnant Ariane, nid d'amours, tondi}\tabularnewline
\hline 
\multirow{2}{0.2\textwidth}{Le peintre de Marcus Lucretius (p. 153) } & Maison de Caecilius Jucundus (V, 1, 26)  & \emph{Jugement de Pâris } & \emph{Hermaphrodite et Silène, Mars et Vénus }\tabularnewline
\cline{2-4} 
 & Maison de Marcus Lucretius  & \emph{Chiron et Achille } & \emph{Poète et acteur, poète et Muse, Pan et Bacchante, Narcisse,
Phrixus et Hellé, Mars et Vénus, Endymion, Cyparissus, Polyphème avec
la lettre de Galatée. }\tabularnewline
\hline 
\multirow{3}{0.2\textwidth}{Le peintre de Méléagre (p. 159) } & Maison de Méléagre (VI, 9, 2)  & \emph{Didon abandonnée, Thétis dans la forge d'Héphaïstos, Jugement
de Pâris, Hector et Pâris, Ganymède} & \emph{Femme assise et amour, Mars et Vénus, Cérès et Mercure, Méléagre
et Atalante, Mars et Vénus, Io et Argus, peintures en relief de stuc,
, jeune fille dînant, Hermaphrodite et Pan, lutte de Pan et Amour,
Hyménée, Ariane abandonnée, Persée et Andromède, Apollon et Daphné }\tabularnewline
\cline{2-4} 
 & Maison du Labyrinthe (VI, 11, 8-10)  & \emph{Pâris et Œnone} & \tabularnewline
\cline{2-4} 
 & bâtiment non identifié  & \emph{Polyphème et lettre de Galatée} & \tabularnewline
\hline 
Le peintre des Noces d'Hercule (p. 165) & Maison des Vettii (VI, 15, 1)  & \emph{Hercule et Augé, Achille à Skyros } & \emph{Figures volantes }\tabularnewline
\hline 
\multirow{2}{0.2\textwidth}{Le peintre de Télèphe (p. 171) } & Maison du Poète Tragique (VI, 8, 3-5)  & Zeus et Héra sur le Mont Ida, Achille rend Briséis, Embarquement de
Chryséis & \tabularnewline
\cline{2-4} 
 & Herculanum Augusteum & Hercule découvrant Télèphe, Chiron et Achille  &  Thésée et le Minotaure, Marsyas et Olympus\tabularnewline
\hline 
\end{longtable}

\part{Objets}
\newcommand{\objet}[2][]{
  \setkeys{bdd}{#1}
  \centering\noindent\includegraphics[width=\textwidth,height=0.5\textheight,keepaspectratio]{#2}

  \noindent \textbf{D\'enomination:} \BDDname\\ 
            \textbf{Sujet:} \BDDsubject\\ 
            \textbf{Mat\'eriau:} \BDDmaterial\\
            \textbf{Datation:} \BDDdate\\ 
            \textbf{Artiste:} \BDDartist\\ 
            \textbf{Lieu de d\'ecouverte:} \BDDdiscovery\\ 
            \textbf{Localisation actuelle:} \BDDlocation\\ 
            \textbf{Cr\'edits photographiques:}\BDDcredits

  \clearpage
}
\input{annexe-objets.tex}

\subsection*{Récapitulatif des tables iliaques et de leurs sujets, d'après l'ouvrage
d'A. Sadurska (Sadurska 1964)}

\begin{longtable}{|>{\centering}p{0.05\textwidth}|>{\centering}p{0.05\textwidth}|>{\centering}p{0.15\textwidth}|>{\centering}p{0.1\textwidth}|>{\centering}p{0.2\textwidth}|>{\centering}p{0.4\textwidth}|}
\hline 
N\textdegree{}  & Réf  & Nom  & Matériau  & Provenance  & Sujet\tabularnewline
\hline 
\hline 
\multicolumn{6}{|c}{Groupe 1}\tabularnewline
\hline 
1 & A & Tabula Iliaca Capitolina & Calcite & Près de la Via Appia & \emph{Iliade, Ilioupersis, Ethiopide, Petite Iliade}; assez complets. \tabularnewline
\hline 
2 & NY & Table Iliaque de New-York & Marbre blanc & Rome & \emph{Iliade} (fragments), \emph{Ilioupersis }\tabularnewline
\hline 
3 & C & Table iliaque \textquotedbl{}Veronensis I\textquotedbl{} & Marbre jaune & Rome & \emph{Cypria, Iliade, Ilioupersis. }\tabularnewline
\hline 
4 & N & Table iliaque \textquotedbl{}bouclier d'Achille\textquotedbl{}  & Giallo antico & Rome & Description du bouclier d'Achille par Homère. \tabularnewline
\hline 
5 & 0 & Table Iliaque, fragment d'un bouclier d'Achille & Palombin Rose  & Rome & Idem \tabularnewline
\hline 
\multicolumn{6}{|c}{Groupe 2}\tabularnewline
\hline 
6 & B & Dessin Sarti.  &  &  & Fragment \emph{Iliade}. Disparu. \tabularnewline
\hline 
7 & Ti & Table iliaque \textquotedbl{}Thierry\textquotedbl{}.  &  &  & 

Fragment \emph{Ilioupersis}. Disparu. \tabularnewline
\hline 
8 & E & Table iliaque dite de Zénodote. & Marbre jaune  & Prov. inconnue & \emph{Ilioupersis }\tabularnewline
\hline 
9 & D & Table iliaque \textquotedbl{}Veronensis II\textquotedbl{} & Marbre blanc jaunâtre & Rome & \emph{Ilioupersis, Iliade }\tabularnewline
\hline 
10 & K & Tabula Borgia & Marbre jaunâtre & Prov. inconnue & Sujet difficile à déterminer.\tabularnewline
\hline 
\multicolumn{6}{|c}{Groupe 3}\tabularnewline
\hline 
11 & H & Table Odysséenne Rondanini & Palombino crème & Prov. inconnue & Hermès, Ulysse, Circé, compagnons d'Ulysse. \tabularnewline
\hline 
12 & F & Table iliaque de la rançon d'Hector & Marbre jaune & Rome? & Troie, la rançon d'Hector. \tabularnewline
\hline 
13 & Ta & Table iliaque Tarentina & Marbre giallo antico & Prov. inconnue & Athéna et Achille, Achille traîne le cadavre d'Hector \tabularnewline
\hline 
14 & G & Table Homérique  & Speckstein & Rome? & Homère, Apollon citharède, Muse, scène de guerre. \tabularnewline
\hline 
15 & Ber & Table iliaque Dressel & Marbre & Rome? & Femme et guerrier. \tabularnewline
\hline 
16 & Sa & Tabula Odysseaca Tomassetti & Marbre blanc  & Rome & 24 scènes de l'\emph{Odyssée }\tabularnewline
\hline 
\multicolumn{6}{|c}{Groupe 4 : ce groupe comporte trois tables à sujet non homérique.}\tabularnewline
\hline 
\end{longtable}

\part{Monnaies}
\newcommand{\monnaie}[2][]{
  \setkeys{bdd}{#1}
  
  \begin{tabular}{p{0.6\textwidth} p{0.4\textwidth}}
    \textbf{R\'ef\'erence:} \BDDref, 
    \textbf{Valeur:} \BDDvalue, 
    \textbf{Atelier mon\'etaire:} \BDDartist\\ 
    \textbf{Magistrat:} \BDDmagistrate, 
    \textbf{Datation:} \BDDdate,
    \textbf{Droit:} \BDDfront, \textbf{Inscription:} \BDDfrontInscr\\
    \textbf{Revers:}\BDDback, \textbf{Inscription:} \BDDbackInscr &
    \raisebox{-0.9\height}{\includegraphics[width=0.4\textwidth]{#2}}
  \end{tabular}
}
\input{annexe-monnaies.tex}

\end{document}
